\subsubsection{Pricing Options by Quasi-Monte Carlo (QMC) with Median Estimator}

Students in this project should have a basic understanding of statistics, such as mean, median, and variance. Calculus I and II are also required. Experience in programming will be preferred.

The sample mean is a good estimator of the population mean. However, the mean is easily affected by the outliers. Pan and Owen proposed a median-of-means approach using randomized digital nets to estimate an integral with a smooth integrand in \cite{SPAMOM22} and obtained an outstanding convergence rate. Inspired by this work, Goda and L'ecuyer studied a new construction-free median QMC rule with lattice points in 
\cite{CFMQMC} and obtained some excellent results. In this project, the students will implement similar ideas to solve the option pricing problem by QMC.

In this project, students will learn how to price options by regular QMC methods and combine them with the median-of-means approach. They will start from the European or Asian options, exercised only at the expiry of the contract. They will make a lot of simulations and comparisons between the traditional and new ways.

SURE students will better understand the differences between mean and median and know how to choose the estimator appropriately. More than that, they will learn and implement the QMC method in other research areas they are interested. The project will also enhance their programming skills through the simulation process. 








