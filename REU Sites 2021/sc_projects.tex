
\subsubsection{Computational Finance with Heterogeneous Big Data}

The purpose of this project in computational finance with heterogeneous big data is to prepare our undergraduate students to become innovative and ambitious members of our society. Successful outcomes would include students applying their knowledge acquired from coursework or independent research reading,  designing data-driven solution approaches, to alleviate important global problems such as financial crisis associated with climate change, harmful large-scale industrial practices, widespread public health, or environmental issues such as an epidemic.  The key components of this SURE project would include real-world (multiple) un/semi-/structured time-series data from finance and other areas; introduction to the literature on existing cutting-edge scientific theories and modern computational tools; as well as an emphasis on rigorous evaluation of analytical results with different metrics. 

One example would be the impact of ESG (environment, sustainability, and government) risk factors on the financial health of companies. There is no doubt that extreme weather events such as floods could wipe out many buildings and infrastructure in the affected areas, leading to a huge financial loss of many companies that have businesses there, possibly resulting in immediate failures to fulfill their financial obligations to business partners and global financial participants who own financial assets issued by these companies. Nevertheless, existing definitions of ESG measures the input data collection process, and scoring methodologies, to name a few, are often very unclear to various stakeholders, if not controversially biased (see, for example, \cite{reiser2019buyer}). In some cases, their connections to company default data are hard to come by. A meaningful project would engage students to critically examine some existing ESG scores (e.g., \cite{friede2015esg}); hypothesize their relationships with more readily available, observable market information such as company share prices or bankruptcy data \cite{pedersen2020responsible, fatemi2018esg}; formulate the problems in mathematics as (stochastic) dynamical systems or optimization; perform statistical analysis, discretization, or extend more powerful explanatory/prediction models of company default such as \cite{jarrow2005default}; and analyze the resultant errors or accuracies. 

In such a project, students would be exposed to important open problems in the emerging area of sustainable finance and become motivated to apply mathematical principles and statistical techniques from courses they had taken. They would come to identify knowledge gaps of their own or the interdisciplinary disciplines and become motivated to improve or even invent new scientific tools and algorithms for solving these challenging problems.
