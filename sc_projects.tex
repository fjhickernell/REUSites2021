
\subsubsection{Computational Finance with Heterogeneous Big Data}

The purpose of this project is to prepare our undergraduates to become innovative and ambitious members of our society via scientific research.   The key components of this SURE project would include real-world un/semi-/structured time-series data from finance and other areas; introduction to the literature on existing or cutting-edge scientific theories; modern computational tools; and emphasis on a rigorous evaluation of analytical results with different metrics. 

An example is the impact of ESG (environment, sustainability, and government) risk factors on corporate  health. There is no doubt that extreme weather events such as floods could wipe out many buildings and infrastructure in the affected areas, leading to a huge financial loss of many companies that have businesses there, possibly resulting in immediate failures to fulfill their financial obligations to business partners and global financial participants who own financial assets issued by these companies. Nevertheless, existing definitions of ESG measures,  input data collection process, and scoring methodologies, to name a few, are often  unclear to various stakeholders, if not controversially biased (see, e.g., \cite{reiser2019buyer}). In some cases, their connections to company default data are hard to come by. A meaningful project would engage students to critically examine some existing ESG scores (e.g., \cite{friede2015esg}); hypothesize their relationships with more readily observable information  such as company share prices or bankruptcy data \cite{pedersen2020responsible, fatemi2018esg}; formulate the problems in mathematics as (stochastic) dynamical systems or optimization; perform statistical analysis, discretization, or extend more powerful explanatory/prediction models of company default such as \cite{jarrow2005default}.

The specific {\bf learning goals} for students are to:
\begin{itemize}[leftmargin=.5cm] \vspace{-1ex}
    \item Describe a selected global problem and potentially new explanatory factors;
   \item  Explain existing quantitative measures of the problem  and data sources for factors;
    \item Hypothesize the relationship of the problem target and factors by performing exploratory data analysis.
\end{itemize}
The specific {\bf research goals} for students are one or more of the following:
\begin{itemize}[leftmargin=.5cm] \vspace{-1ex}
    \item Build prediction models of the problem in relation to the risk factors, following, for example, the Cross-Industry Standard Process for Data Mining (CRISP-DM);
    \item Apply or enhance existing  algorithms from established software for  problem solving;
    \item Measure and report on proper model errors;
    \item Identify shortcomings or questions of their own work for future research.
\end{itemize}